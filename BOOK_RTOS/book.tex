\documentclass[11pt,a4paper]{book}

\usepackage{graphicx}
\usepackage{listings}
\usepackage{float}
\title{An Introductory Guide to FreeRTOS on LPC2148}
\usepackage{graphicx}

\author{e-Yantra Team}
\date{\today}


\usepackage[english]{babel}
\usepackage{blindtext}
\usepackage{graphicx}
\usepackage{listings}
\usepackage{graphics}
%packagr for hyperlinks
\usepackage{hyperref}
\hypersetup{
	colorlinks=true,
	linkcolor=blue,
	filecolor=magenta,      
	urlcolor=cyan,
}

\urlstyle{same}
%use of package fancy header
\usepackage{fancyhdr}
\setlength\headheight{26pt}
\fancyhf{}
%\rhead{\includegraphics[width=1cm]{logo}}
\lhead{\rightmark}
\rhead{\includegraphics[width=1cm]{logo}}
\fancyfoot[RE, RO]{\thepage}
\fancyfoot[CE, CO]{\href{http://www.e-yantra.org}{www.e-yantra.org}}

\pagestyle{fancy}

%use of package for section title formatting
\usepackage{titlesec}
\titleformat{\chapter}
{\Large\bfseries} % format
{}                % label
{0pt}             % sep
{\huge}           % before-code

%use of package tcolorbox for colorful textbox
\usepackage[most]{tcolorbox}
\tcbset{colback=cyan!5!white,colframe=cyan!75!black,halign title = flush center}

\newtcolorbox{mybox}[1]{colback=cyan!5!white,
	colframe=cyan!75!black,fonttitle=\bfseries,
	title=\textbf{\Large{#1}}}

%use of package marginnote for notes in margin
\usepackage{marginnote}

%use of packgage watermark for pages
%\usepackage{draftwatermark}
%\SetWatermarkText{\includegraphics{logo}}
\usepackage[scale=2,opacity=0.1,angle=0]{background}
\backgroundsetup{
	contents={\includegraphics{logo}}
}

%use of newcommand for keywords color
\usepackage{xcolor}
\newcommand{\keyword}[1]{\textcolor{red}{\textbf{#1}}}

%package for inserting pictures
\usepackage{graphicx}

%package for highlighting
\usepackage{color,soul}

%new command for table
\newcommand{\head}[1]{\textnormal{\textbf{#1}}}


\begin{document}
	\maketitle
	\newpage
	\tableofcontents
	\newpage
	\section{Introduction to RTOS}
	\subsection{What is RTOS ?}
	"RTOS" stands for Real-Time Operating System.
	It is a type of operating system used for real time applications in embedded systems.
	RTOS is known for its characteristics that helps is in many applications.
	\begin{itemize}
		
		\item Reliability :\\
		RTOS provides more reliability as compared to GPOS.
		It has more control over events in real time and they are always available to provide service.
		Some systems are required to run for a longer period of time without human intervention, for these purposes RTOS can be very useful.
		
		\item Determinism:\\
		RTOS entirely functions over deadlines which makes            it more efficient.
		It means that for each process a specifc deadline or a time-period is specified within which it has to finish that particular process.
		
		\item Scheduling:\\
		In this operating system, user has more control over scheduling a particular task or a process depending on its priority.
		So we can define the priority for that particular  task and also the frequency with which it should occur ( more like a delay).
		In GPOS all the scheduling functions are process based and user has less control on them.
		Task defined in RTOS are preemptive. 
		Generally, in an operating systems there are two types of tasks viz. High priority tasks and Low priority tasks.
		High priority task can meet their deadlines consistently because of the preemptive property.
		
		\item Scalability:\\
		RTOS is used in wide variety of applications in the field of embedded systems.
		So it is scalable depending on the application requirements (i.e we can add or remove modular components depending on our use).
	\end{itemize}
	
	\subsection{Types of tasks in RTOS}
	\begin{enumerate}
		
		
		\item\textbf{Hard real-time tasks:}\\
		These types of tasks strictly run based on deadlines.
		If a particular task is not finished within the predetermined deadlines  then the system is considered to be a failed system.
		Applications : Anti-missile systems, Air bag mechanisms etc.
		
		\item\textbf{Firm real-time tasks:}\\
		Similar  to hard real-time tasks they should also meet the deadlines.
		But if they don't meet then that doesn't make this a failed system, but the results that are produced after the deadlines are discarded and the utility of the system becomes zero.
		Applications : Multimedia
		
		\item\textbf{Soft real-time systems:}\\
		Here the deadlines are not expressed as some absolute value but they are expressed as a average response time required by the task.
		If the task is finished then the utility of the task is 100%. But if they fail to meet them, then the utility of the system gradually falls depending on the extra time that is taken past the deadline.  
	\end{enumerate}
\newpage	
	\section{What is FreeRTOS}
	
	For statrters FreeRTOS is just a bunch of C files which enables us to implement RTOS in around 32 microcontrollers. FreeRTOS provides files which can be used in multiple microcontrollers with some microcontroller specific support files.
	\\
	\\   
	The main advantage of Implementing FreeRTOS in any microcontroller is the ability to multi-task.
	\\
	\\
	MultiTasking enables the device to execute multiple "Tasks" at the same time.
	MultiTasking in single core systems is implemented by allocating each task a time slice of the processor, In this way Multiple Tasks can be executed at the "same time".
	\\
	\\
	\subsection{Advantage of FreeRTOS}
	\begin{enumerate}
		\item Proper Utilization of Resources.
		\item Low foot-print.
		\item Priority based scheduling of Tasks.
		\item API's for Semaphores.
		\item API's for Making and managing queues.
	\end{enumerate} 

	\subsection{Drawbacks/Disadvantages (write some fancy word)}
	\begin{enumerate}
		\item Use of TaskNotification to implement MailBox.
		\item Not readily Portable to all devices.
		\item Limited source material.
	\end{enumerate} 
\newpage
\section{Requirement}
\begin{enumerate}
	\item Knowledge of C++ 
	\item FreeRTOS source files/API
	\item Keil compiler
	\item Flash magic
	\item FireBird V (LPC2148)
\end{enumerate}

\section{Setting up the environment}

\subsection{Installing the softwares}
There are two softwares that we need to install before proceeding.
\begin{itemize}
	\item Keil uVision4 IDE.
	\item Flash Magic.
\end{itemize}
After the installation we need to download the FreeRTOS library files from the website of www.freertos.org. During this period we will be using the FreeRTOSv9.0.0 version for the library files.

\subsection{Creating a new project}
After installing all these softwares and downloading the FreeRTOS file,follow the following steps to create a new project.
\begin{enumerate}
	\item Open Keil uvision4 and select 'project' and then select new project.
	\item Select the NXP option in the window and select lpc2148 processor.
	\item Then go to file and select new and in the window that opens start coding.
	\item But before compiling we need to include certain files and libraries.

Follow the following steps to include those files.
\begin{itemize}
	\item Right click on the target folder that is created on the left hand side of the page.
	\item Select the 'options for target' option and set the settings as per the following images.
	\begin{figure}[h]
		\centering
		\includegraphics[width=10cm,height=6cm]{Capture.PNG}
		\caption{Create HEX}
	\end{figure}
\item Go to the c++ option on the same window. On that same page we will see a include paths option. Open that to include the files as shown in the image.
\begin{figure}[H]
	\centering
	\includegraphics[width=10cm,height=6cm]{c++.PNG}
	\caption{C++ configurations}
\end{figure}
\begin{figure}[H]
	\centering
	\includegraphics[width=10cm,height=6cm]{c++foldersetup.PNG}
	\caption{C++ Folder Setup}
\end{figure}
\item After this go to the ASM page which is right next to the C++ page and again open the include paths page and include the following.
\begin{figure}[H]
	\centering
	\includegraphics[width=10cm,height=6cm]{asm.PNG}
	\caption{Include paths}
\end{figure}
\begin{figure}[H]
	\centering
	\includegraphics[width=10cm,height=6cm]{asmpath.PNG}
	\caption{ASM Configurations}
\end{figure}

\item Go to the linker option and check the following settings.
\begin{figure}[H]
	\centering
	\includegraphics[width=10cm,height=6cm]{linker.PNG}
	\caption{Include Files}
\end{figure}
Now lets begin including files to our project.
\item Firstly double click on the cource folder that is generated below the target folder.If it contains the 'startup.s' file then delete it.Include the new startup.s. This file is located in \\"FreeRTOSv9.0.0/FreeRTOS/Demo/ARM7LPC2129KeilRVDS" this folder.
\item Add the main.c file which contains our program.
\item Add the "tasks.c", "list.c" and "queue.c" from the source folder which is inside the FreeRTOSv9.0.0 folder which we downloaded.
\item Add the "FreeRTOS.h" and "freertosconfig.h" from include folder which is located inside the source folder.
\item Now go inside the portable folder inside the source folder and go to RVDS folder and open the ARMLPC21xx folder and add the "port.c" and "portASM.s".
\item And lastly inside the portable folder go to the "MemMang" folder and include the "heap2.c".
\item You can include the "LCD.c" from the experiments folder of FireBird V LPC2148 folder.\\
Your libraries should now consist of all these files.
\begin{figure}[H]
	\centering
	\includegraphics[width=7cm,height=10cm]{libraries.PNG}
	\caption{LIBRARIES}
\end{figure}

\end{itemize}

\end{enumerate}
\section{Getting Started !}

	\subsection{DataTypes}
		FreeRTOS defines counterparts of few basic data types
		\\
		\\
\begin{tabular}{|l|l|}
	\hline
	\textbf{Data Type }&\textbf{General Data Type}\\
	 \hline
	 	portCHAR&char\\ \hline
		portSHORT&short\\ \hline
		portLONG&long\\ \hline
		portTickType&This is used to store the tick count\\ \hline
		portBASE\_TYPE&Generally used for Bool type data,is 32 bit for 32 bit type uC\\ \hline
\end{tabular}

	\subsection{Variable Names}
	The Data type is prefixed to the name of a variable for e.g \\
	
	In vTaskDelay "v" denotes the return type "void".\\
	
	In xTaskCreate "x" denotes portBASE\_TYPE.
	
	\subsection{Macros}
	Refer to page 168-169 of the RTOS document by Richard Barry.
	
	\subsection{Creating Tasks}
	
	\begin{lstlisting}
	BaseType_t xTaskCreate(    TaskFunction_t pvTaskCode,
	const char * const pcName,
	unsigned short usStackDepth,
	void *pvParameters,
	UBaseType_t uxPriority,
	TaskHandle_t *pxCreatedTask
	);
	\end{lstlisting}
	
	\begin{itemize}
		\item BaseType\_t :Can be used to check if the task has been created or not.If the returned value is pdTRUE the task has been created if pdFALSE is returned the task was not created.
		
		\item  pvTaskCode:This parameter is a pointer to the task which has been created.
		
		\item pcName :Name given to a Task created so that user can easily identify a task,this parameter enables the programmer to easily identify a task.
		
		\item usStackDepth :The amount of memory/space which a given task is to be allocated is passed as a parameter through this value.
		
		\item uxPriority :Each task is assigned a priority on the basis of which it is allocated the processor time.priority assigned are natural numbers ,as the value of number increases priority increases.
		
		\item pxCreatedTask :Tasks are assigned handles using which they can be referred by other tasks. 
	\end{itemize}
	 
	 Same Task can have multiple instances by varying the priority,parameters passed ,pcName.
	
	\subsection{Frequently used API's}
	\begin{itemize}
		\item vTaskDelay :Takes the Clock Ticks as parameter and suspends the Task for those many cycles. e.g vTaskDelay(1000);
		
		\item vTaskSuspend :Takes Taskhandle as a parameter and Suspends the "passed" task indefinitely.
		e.g. vTaskSuspend(t1) : suspends task t1 
		     vTaskSuspend(NULL) : suspends running task
		
		\item vTaskResume :Also Takes Taskhandle as a parameter resumes the Task from suspended state
		e.g. vTaskResumed(t1) : resumes task t1 
		
		\item tskIDLE\_PRIORITY: Priority of the idle task,used to fix priority of Tasks created.
	\end{itemize}
	\newpage
	
	\section{MultiTasking}
	Multitasking is running multiple processes at the same time.
	In a multi-processor system it implies that each core of processor is executing different tasks i.e. multiple tasks at the same time.
	Whereas in a single processor system the operating system schedules tasks in such a way that all the tasks are performed simultaneously i.e. each task gets a limited amount of Processor time, after the time expires the running task is suspended and another task is executed. The original task gets the resources again when all the tasks are given equal amount of processor time. 
	
	\subsection{Code :}
	\lstinputlisting[language=C]{rtos_multitasking.c}
	
	\subsection{Explanation}
	The given code is used to create 3 tasks 
	\\
	\\
	1st task switches on the buzeer then the task is suspend and after a while the task is resumed and buzzer is switched off.
	\\
	\\
	2nd task is a motion task which aims to give the bot a forward motion.
	\\
	\\
	3rd task prints the value of a counter on the LCD,the value of counter resets when it reaches 100.
	\\
	\\
	The statements for the task which are to be executed are placed inside an infinite loop so that they can be continiously executed.
	\\
	\\
	the xTaskCreate statement "creates tasks" this can be thought of as function call statements which call the respective functions.
	\\ 
	\\
	The vTaskStartScheduler starts scheduling the tasks i.e. allocating processor to the tasks.
	
	If the tasks are not placed in the infinite loop the statements are executed ones and the task is completed.
	
	The ouput can be observed by uploading the code to LPC2148 based FBV.
	  
	  \newpage
	\section{Introduction to semaphore}
	There are a limited number of resources available to any system,Similarly any microcontroller has a limited number  resources available.
	\\ \\
	As the complexity of the application Increases the number of Tasks running also Increases,more and more Tasks compete for the available Processor time or The I/O devices available.
	\\ \\
	To ensure equal availability of resources to all the Tasks Operating Systems provide a facilities through semaphores.
	\\ 
	The Greek word sema means sign or signal, and -phore means carrier . So Semaphore = signalling.
	\\
	Semaphores can be classified into
	\\ 
	\begin{itemize}
	\item Binary Semaphores
	\item Mutex	 
	\item Counting Semaphores
	
	
	\end{itemize}
	\subsection{Binary Semaphores}
	
	Binary semaphores are used for Task synchronisation.
	If a process ocuppies a resource the value of Binary semaphore is 1 else 0 i.e it gives information only if the resource is available or not.
	
	\subsection{Mutex}
	
	Mutex stands for Mutual Exclusion.Any Task which requires a resource can "Block" the resource.when the Task uses the resource it can "Give" the resource.
	
	\subsection{Counting Semaphore}
	
	Counting semaphores are used to count resources and keep track of Multiple resources.
	\\
	 
	\subsection{Mutex vs Binary Semaphore}
	\begin{itemize}
		\item Mutexes are used for Resource Protection from other tasks//processes whereas Binary semaphores are used for task synchronistaion
		\\
		\item It is the responisibility of the occupying function to release the mutex,but a binary semaphore can be released even from ISR or any other functions.
		\\
		\item On the implementation level it is the Responibility of the Coder to ensure that the Mutex is only given by the task which takes it.
		
	\end{itemize}
		
	
	
	
	\newpage	
	\section{Binary Semaphore}
	
\subsection{Code : }
	\lstinputlisting[language=c]{bin.c.}	
	\newpage
\subsection{Explanation}
	\begin{itemize}
	\item \textbf{Variable declaration}	
	
	\begin{lstlisting}
	SemaphoreHandle_t xSemaphore;
	\end{lstlisting}
	
	 This statement declares a variable of type "SemaphoreHandle\_t"
	
	\item \textbf{Creation of the semaphore}
	
	\begin{lstlisting}
	xSemaphore=xSemaphoreCreateBinary( );
	\end{lstlisting}

\item \textbf{Working of code}
	  \\
	  \\
	  The forward function Waits for portMAX\_DELAY i.e for maximum amount of time so that the control of Resources is available.
	  \\
	  \\
	  Similarly the back function waits for maximum time to get access to the resources.
	  \\
	  \\	
	  As soon as execution of Tasks starts the resources are occupied by the back function(vTaskDelay restricts forward function),The control\_switcher function is suspended for 1200 clock counts and Gives away the semaphore.
	  \\
	  \\
	  As soon as the semaphore is released the forward function waiting for allocation of resources occupies them,the cycle continues with control\_switcher releasing the semaphore.  
		\\
		\\
		\\
\item \textbf{Serial monitor Output} 
\\
\\
\includegraphics[width=10cm]{bin}

\end{itemize}
\newpage 
\section{Mutex}

\subsection{Code : }
\lstinputlisting[language=c]{mut.c.}	
\newpage

\subsection{Explanation}
\begin{itemize}
	\item \textbf{Variable declaration}	
	
	\begin{lstlisting}
	SemaphoreHandle_t xSemaphore;
	\end{lstlisting}
	
	This statement declares a variable of type "SemaphoreHandle\_t"
	
	\item \textbf{Creation of Mutex}
	
	\begin{lstlisting}
	xSemaphore = xSemaphoreCreateMutex();
	\end{lstlisting}
	
	\item \textbf{Working of code}
	\\
	\\
	There are Two Tasks forward and back, when executed
	\\
	\\
	The forward function Waits for 1000 clock cycles for the resources,In case the resources are not available the Task sends a message about The lack of availability of resources.
		Similarly the back function waits for same amount of time for resources.
	\\
	\\	
	As soon as execution of Tasks starts the resources are occupied by one of the the task and that task blocks the acess of those resources through a mutex.
	\\
	\\
The task executes and when the execution is completed it "Gives" the Mutex and therefore the releases the resources,another waiting task then occupies those resources and blocks for a period of time it requires.
	\\
	\\
	\\
	\item \textbf{Serial monitor Output} 
	\\
	\\
	\includegraphics[width=8.5cm]{mut}
\end{itemize}

\newpage 

\section{Counting Semaphore }

\textbf{:Implemented by dining Philosophers Problem}
\subsection{Code : }
\lstinputlisting[language=c]{cont.c.}	
\newpage

\subsection{Explanation}
\begin{itemize}
	\item \textbf{Variable declaration}	
	
	\begin{lstlisting}
	SemaphoreHandle_t xSemaphore;
	\end{lstlisting}
	
	This statement declares a variable of type "SemaphoreHandle\_t"
	
	\item \textbf{Creation of Counting semaphore}
	
	\begin{lstlisting}
xSemaphore = xSemaphoreCreateCounting( 5, 5 );
	\end{lstlisting}
	
	Here 1st parameter gives the maximum count and 2nd parameter is the initial count.
	If the semaphore is used for counting events 2nd parameter would be 0 and if used for resources management it would be equal to maximum or initial count.
	\\
	\item \textbf{Task Creation }
	\begin{lstlisting}
	xTaskCreate(vfork,"Philospher 1", 300 ,"P1",
	 tskIDLE_PRIORITY + 1, NULL);
	.
	.
	\end{lstlisting}
	Here vfork is a single Task which on variation of Parameter P1,P2...etc behaves as a different task,ecah task has its own stack and act as if they are independent.All the tasks have same priority and get equal time at the processor.
	\\
	\item \textbf{Working of code}
	
	The Tasks created are by changing the parameters of a single task.
	\\
	\\
	When each time a "Philosopher" is allocated the processor time it checks for the number of available "Forks".If the forks are available and then check for the Right fork and the philosopher "picks up the left fork" then when the "Philosopher" again gains the processor time it waits for Left fork to be available and proceeds to eat.
	
	when 5 "Philosophers" are allocated simulatenously the semaphore keeps track of the available forks .
	
	
	\newpage
	\item \textbf{Serial monitor output}
	\\
	\includegraphics[width=12cm]{cont}
\end{itemize}
\newpage


\section{Task Notification}
	There occurs instances when tasks needs to communicate with each other.Semaphores are one of the methods by which tasks communicate with each other.Two other methods by which tasks communicate with each other are 
	\begin{enumerate}
		\item MailBox
		\item Queues
	\end{enumerate}
	
   Tasks in mailbox communicate by sending "Mails" to each other.In FreeRTOS mailbox is implemented by Task Notification.
   
   Each Task has an associated notification value using which they can be "notified".When a task is notified, Task notifications can update the receiving task's notification value in the following ways:
   \begin{itemize}
   	 
\item Set the receiving task's notification value without overwriting a previous value
\item Overwrite the receiving task's notification value
\item Set one or more bits in the receiving task's notification value
\item Increment the receiving task's notification value 
   
   \end{itemize}
   \newpage
   \subsection{Code:}
   \lstinputlisting[language=c]{mbox.c.}	
   \newpage
   
   \subsection{Explanation}
   
   Above is a simple code which has four tasks(vn1,vn2,vn3,vn4) which notify a 5th task 'noticier',The 5th task prints which task notified it. 
   \\
   \\
   \begin{itemize}
   	\item \textbf{xTaskNotify()}
   	This function is used to notify other tasks general format is as specified below 
   		\begin{lstlisting}
   		xTaskNotify(xHandle, 0x03, eSetBits);
   		\\(Task Handle,Notification value,eAction)
   		\end{lstlisting}
  
  
   	\textbf{Parameters}
   	\begin{itemize}
   		\item \textbf{Task handle}:The handle of the task which needs to be notified.
   		\item \textbf{Notification value}:Value used for notification
   		\item \textbf{eAction}:The type of action which is to be carried out upon the specified task.The types are as specified below:
		   		\begin{itemize}
		   			\item eNoAction :The Task receives the value but no action takes place,can used to Resume a suspened task.
		   			
		   			\item eSetBits :The existing Notification value will be Bitwise OR-ed with the Notified value to obtain a new value.
		   			
		   			\item eIncrement :Increments the existing value.
		   			
		   			\item eSetValueWithOverwrite :Overwrites the existing Notification value.
		   			
		   			%\item eSetValueWithoutOverwrite :Didn't understand this part
		   			
		   		\end{itemize}
		   		
   	\end{itemize}
   	
   	\textbf{Return value :}
   	  	Returns pdTRUE if Task has been Notified else pdFALSE.
   	
   	\item \textbf{xTaskNotifyWait() } The Function waits to receive a Notification and has parameters which govern the actions upon the received data.
   	\begin{lstlisting}
   	xTaskNotifyWait( 0x00,0xffff,&ulNotifiedValue,1000 )
   	\\(clear bits on entry,clear bits on exit,notified value,time out)
   	\end{lstlisting}
   	\textbf{Parameters}
   	\begin{itemize}
   		\item \textbf{ulBitsToClearOnEntry}:Specifies the Bit position which needs to be cleared as soon as the Notification is received.
   	    
   	    \item \textbf{ulBitsToClearOnExit }:Specifies the Bit position which needs to be cleared before xTaskNotifyWait() function exits if a notification was received
   	    
   	    \item \textbf{pulNotificationValue}:The Notification value before exit is taken and stored in this.
   	       
   	    \item \textbf{xTicksToWait  } :This specifies the timeout period for which the function call waits for a notification.
   \end{itemize}
   
   \item \textbf{Working}
   \newline
   Tasks vn1,vn2,vn3,vn4 With different frequencies send a "message" to a noticer task through a hex value,these hex values are compared to find out which task sent the message.
   \\
   For the first few ticks no task is sending a notification so the noticier prints a "No Notice" message,as It starts receiving messsages it starts acknowledging the received messages. 
   \end{itemize}
   
   \includegraphics[width=12cm]{mbox}
	\newpage
	\section{Queue}
	 \subsection{Intro}
	 In RTOS, inter-process communication is possible. It means that you can communicate between two task and control them on the basis of this communication.\\ But first we need to understand \textbf {what is queue?}\\
	Consider a dynamic buffer. Dynamic in the sense of memory allocation. We can allocate the size of the buffer as per our requirements. There are two things that we can change, one is the size of each data the buffer can carry and other is the number of data the buffer can send with each data of the size defined by us. this buffer is known as queue.\\
	
	\subsection{Code}
	\lstinputlisting[language=c]{q.c.}
	
	\subsection{Explanation}
	\begin{itemize}
		\item \textbf{QueueHandle\_t} :A predefined data type used to reference a queue.
		
		\begin{lstlisting}
		QueueHandle_t xQueue= 0;
		\end{lstlisting}	
		
		\item \textbf{xQueueCreate(a,b)}:Creates a queue of 'a' continuous memory locations,where each memory location is of 'b' bytes each.It returns a 'pointer' to the queue which is stored in the queuehandle.
		
		\begin{lstlisting}
		xQueue = xQueueCreate(7,40);
		\end{lstlisting}
		
		\item \textbf{xQueueSend(QueueHandle,data,timeout period)}:This function is used to send data to queue,It has three parameters
		
		\begin{enumerate}
			\item\textbf{QueueHandle}:It gives the address of the queue in which data has to be stored.
			
			\item\textbf{Data}:The data which needs to be stored in the queue.
			
			\item\textbf{Timeout period}:This specifies the amount of time for which the function waits if the queue is unavailable(i.e data is being sent to queue or queue is full) 
			\newline		
		\end{enumerate}
		
		\textbf{Return value :}Returns pdTRUE if Data has been sent to Queue else pdFALSE e.g..
		\begin{lstlisting}
		if(xQueueSend(xQueue,p,1000) == pdTRUE)
		...
		\end{lstlisting}
	\item \textbf{xQueueReceive(QueueHandle,data,timeout period):}This function is used to receive data from a queue,It has three parameters
	
	\begin{enumerate}
		\item\textbf{QueueHandle}:It gives the address of the queue from which data has to be obtained.
		
		\item\textbf{Data}:A variable in which the poped data has to be stored.
		
		\item\textbf{Timeout period}:This specifies the maximum amount of time for which the function waits if the queue is unavailable(i.e data is being received by another task or queue is empty) 
		\newline		
	\end{enumerate}
	\textbf{Return value :}Returns pdTRUE if Data has been Received from the Queue, pdFALSE id queue is empty e.g..
	\begin{lstlisting}
	if(xQueueReceive(xQueue,rx,1000) == pdTRUE)
	...
	\end{lstlisting}
	
	\item\textbf{Working of code:}
	Initially a queue is created which can accomodate 7 elements in which each of them can occupy 40 Bytes.
	
	There are 4 tasks which send data to the queue and one task which receives data from the queue.
	
	As the tasks are created data is pushed into the queue and the receiving task is resumed which inturn pops the data and prints them through the serial comm port.
	
	From the Output screenshots it can be observed how  data pushed 1st is popped out 1st (Queue mechanism).  
	\end{itemize}
	\includegraphics[width=12cm]{q}
	
	\newpage
	
	\section{Context Switching}
	
	\subsection{Intro}
	Consider two process A and B. A is a process that is currently scheduled to run in the CPU as soon as it is allotted with CPU time. Once it is allotted with CPU time it will start executing. Suppose that B is process or thread that is at a higher priority than process A. Then the CPU time will be allotted to process B and process A will be pre-empted/suspended from the time span and CPU time will be given to process B. When process A is pre-empted/suspended then the flags and the register values are stored somewhere in the memory of that particular process. Once process B has finished execution then process will resume from the moment where it was pre-empted/suspended. This can be done because process A was able to store its values. This is called as context switching. 
	
	\subsection{Code}
	
	
	%\item Use the following code to do Context %Switching.\vspace{2mm}
	\lstinputlisting[language=C]{cswitch.c}
	%\item\textbf{Code Explanation:}
	
	\newpage
	
	\subsection{Explanation}
	\textbf{xTaskCreate:} Three tasks are created viz. Counter, motion and lcd printing. All these tasks have different priorities as seen in the program. Highest priority is given to the LCD task. It will be scheduled first and it will not give the CPU time until its finished.\\
	Now, in the three tasks the vCounting task is just a counter, vmotion is a task that will always keep the robot moving in the forward motion and lastly the vlcdprint will print for a period of time and then it will be interrupted. While its execution its priority is reduced than other tasks. The priority changing is done by the function called as \textbf{vTaskPrioritySet}. In this function we give two parameters viz. the task handle and the priority. From the code we can see that the priority for task 3 is changed from 4 to 1. Hence, task 3 now has the lower priority so this task is now is pre-empted and motion task has the higher priority so it is executed first. After that this pre-empted task will resume from the point where the task was pre-empted. This will be clearly seen on the lcd.  
	
	\section{Application Based Experiments}
	\subsection{State collection}
	State collection involves storing data of sensors at each instance.The advantage of having the state of robot(i.e sensor values) is that it would help in debugging or simulating an already conducted experiment.
	
	The experiment given below would collect sensor data,store it in a string and send it via serial port every 100ms and there is a corresponding python script which would store the collected data with the time stamp in a script file.
	\\
	Each sensor data is seperated by a ',' and each set of data is seperated by delemiters ',00,255,'
	\subsubsection{State collection Code}
	\lstinputlisting[language=c]{statecol.c.}
	\newpage 
	\subsubsection{Python script}
	\lstinputlisting[language=python]{serialcomm.py.}
	
	\subsubsection{Sample output}
	\includegraphics[width=14cm]{serialcomm}
	
	\section{Collision Avoidance}
	\subsection{Theory and Description}
	In this section we will study about using all the IR that are connected to master and the slaves of Firebird V LPC2148. There are two methods to setup communication between a master and a slave viz. I2C(pronounced as I squared C) and SPI(Serial Peripheral Interface). In LPC2148 the master and slave are connected through SPI communication. There are two SPI's available in LPC2148 viz. SPI0 and SPI1. Here the master and slave1 are connected through SPI1 and the two slaves are communicating with each other through UART.\\
	\newpage
	\textbf{SPI COMMUNICATION:}\\
	\begin{figure}[h]
		\centering
		\includegraphics[width=10cm,height=5cm]{SPI.PNG}
		\caption{ONE MASTER AND SLAVE}
	\end{figure}
	
	In SPI communication master always initiates the communication. It generates the clock signal to initialize the communication. There is a slave select/chip select line which goes low in order to select a particular slave. Because in SPI communication there can be multiple slaves but there is only one master to control them.\\
	SPI uses separate lines for data and a clock that keeps both sides in perfect synchronization. The clock is an oscillating signal that tells the receiver exactly when to sample the bits on the data line. This could be the rising  or falling edge of the clock signal.\\ 
	
	Master puts data on MOSI(Master Out / Slave In) line with clock(SCLK), When the receiver detects the rising edge, it will immediately look at the data line to read the next bit.\\ 
	
	When data is sent from the master to a slave, it’s sent on a data line called MOSI(Master Out / Slave In). If the slave needs to send a response back to the master,  slave will put the data onto a third data line called MISO(Master In / Slave Out) and master continue to generate clock. \begin{figure}[h]
		\centering
		\includegraphics[width=10cm,height=5cm]{multiplespi.PNG}
		\caption{ONE MASTER MULTIPLE SLAVES}
	\end{figure}
	\newpage
	\subsection{Experiment}
	Following is the code for collision avoidance:\\
	\lstinputlisting[language=C]{coli.c}
	\newpage
	In this code there are total 13 tasks. Out of those one is especially for forward motion and the other is for receiving the data from the slave and assigning it to a particular variable. There are 8 tasks in the code each of them is for a particular IR. In that task it determines the movement of the robot once it encounters any obstacles. For eg: consider IR 3 which is the front sharp sensor. There can be 3 possibilities when it faces an obstacle. It can go left or right or go back. In the task of "IR3" first that task will check for sensor 5 if there is no obstacle the it will turn right. Suppose if there are some obstacles there too then it will turn left. And if that is also blocked then it will just go back. Similarly, for each and every sensor there are such cases. There is a task called as " lcdprint" that will store the values of the sensors. In order to remove some of the blind spots (like the space between the two sensors) some more tasks are created like 'IR2-IR3' and 'IR3-IR4' which evaluates the obstacles for that spaces. There is a task that will evaluate obstacles in certain depth like the corner of a square. It is represented by 'IR2-IR3-I4' which considers the obstacles detected by all the three sensors and evaluates accordingly and adjusts the motion of the robot.\\
	This is the working of the robot.
	
\newpage

	\section{References}
	%\begin{enumerate}
	%\item  http://www.rtos.be/2013/05/mutexes-and-semaphores-two-concepts-for-two-different-use-cases/
	
	%\item http://www.ocfreaks.com/cat/embedded/lpc2148-tutorials/
	%\item http://www.freertos.org/Inter-Task-Communication.html
	%\item http://tinymicros.com/
	%\item http://www.profdong.com/elc4438\_spring2016/\\USINGTHEFREERTOSREALTIMEKERNEL.pdf
	%\end{enumerate}	
	
	\begin{itemize}
		\item http://tinymicros.com/
		\item Timer:https://d1b10bmlvqabco.cloudfront.net/attach/io19jz7pw8d4qk/hxg3z49l9je2dc/ipft11xb342g/Timer\_interrupt.pdf
		
		\item ucos:http://www.soe.uoguelph.ca/webfiles/engg4420/uCOS-II\%20The\%20Real-Time\%20Kernel.pdf
		
		\item Datasheet:http://www.keil.com/dd/docs/datashts/philips/lpc2141\_42\_44\_46\_48.pdf
		\item Which to choose:http://www.edaboard.com/thread72908.html
		\item Rtos basic:http://www.cs.sfu.ca/CourseCentral/300/oneill/Resources/Segment7.pdf
		\item Git already done:https://github.com/selste/openDrive
		\item Keil kernel:http://www.keil.com/rl-arm/kernel.asp
		\item ARM porting:http://www.nxp.com/documents/selection\_guide/75016983.pdf
		\item \textbf{BOOK :}
		http://www.profdong.com/elc4438\_spring2016/USINGTHEFREERTOSREALTIMEKERNEL.pdf
		\item http://www.ocfreaks.com/lpc2148-uart-programming-tutorial/
		\item http://www.ocfreaks.com/cat/embedded/lpc2148-tutorials/
		\item https://www.pantechsolutions.net/microcontroller-boards/uart-interfacing-with-lpc2148-arm7-primer
		\item http://www.rtos.be/2013/05/mutexes-and-semaphores-two-concepts-for-two-different-use-cases/
		\item https://sites.google.com/site/rtosmifmim/home
		\item http://www.keil.com/dd/docs/datashts/philips/user\_manual\_lpc214x.pdf
		\item https://learn.sparkfun.com/tutorials/exploring-xbees-and-xctu
		\item http://alumni.cs.ucr.edu/~amitra/sdcard/Additional/sdcard\_appnote\_foust.pdf
		\item https://www.youtube.com/watch?v=FsHVBxXo45Y
		\item http://www.nxp.com/documents/application\_note/AN10736.pdf (VI)
		\item https://www.element14.com/community/docs/DOC-61511/l/arm-lpc2148-usb-hid-human-interface-device-example
		\item https://www.youtube.com/playlist?list=PL0E131A78ABFBFDD0
		
		\item \textbf{Camera}
		\begin{itemize}
			\item	http://embeddedprogrammer.blogspot.in/2012/07/hacking-ov7670-camera-module-sccb-cheat.html
			\item http://forum.arduino.cc/index.php?topic=125767.0
			\item	http://forum.arduino.cc/index.php/topic,159557.0.html 
		\end{itemize}
		
		\item \textbf{USB}
		\begin{itemize}
			\item	http://www.keil.com/forum/18912/usb-interface-with-lpc2148/
			\item	http://www.keil.com/forum/15398/usb-hid-mouse-driver-using-lpc2148/
			\item	http://www.nxp.com/documents/application\_note/AN10736.pdf
			\item	https://www.sparkfun.com/tutorials/94
			\item	https://community.arm.com/thread/6323
			
		\end{itemize}
		
		\item \textbf{SPI}
		\begin{itemize}
			\item http://www.datasheetarchive.com/dlmain/Datasheets-NXP/DSANXP010006749.pdf
			
			%\item http://siwawi.bauing.uni-kl.de/avr\_projects/arm\_projects/arm\_memcards/#chanfat\_lpc2k\_spi
		\end{itemize}
		
		
		
		\item \textbf{Sharp sensor Equation :}
		http://ediy.com.my/blog/item/92-sharp-gp2y0a21-ir-distance-sensors
		
		\item \textbf{Wifi module}
		https://www.pantechsolutions.net/media/k2/attachments/Wi\_Fi\_Interfacing\_With\_ARM\_Primer.pdf
		
		\item \textbf{Pyserial}
		https://github.com/gskielian/Arduino-DataLogging/blob/master/PySerial/README.md
		
		\item \textbf{RPI}
		https://www.raspberrypi.org/forums/viewtopic.php?f=26\&t=21610
		
	\end{itemize}
	
	\end{document}


